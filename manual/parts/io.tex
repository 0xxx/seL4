%
% Copyright 2014, General Dynamics C4 Systems
%
% This software may be distributed and modified according to the terms of
% the GNU General Public License version 2. Note that NO WARRANTY is provided.
% See "LICENSE_GPLv2.txt" for details.
%
% @TAG(GD_GPL)
%

\chapter{\label{ch:io}Hardware I/O}

\section{Interrupt Delivery}
\label{sec:interrupts}

Interrupts are delivered as notifications. A thread
may configure the kernel to signal a particular \obj{Notification}
object each time a certain interrupt triggers. Threads may then wait for
interrupts to occur by calling \apifunc{seL4\_Wait}{sel4_wait} or
\apifunc{seL4\_Poll}{sel4_poll} on
that \obj{Notification}. In the notification word returned from
either call,  bit $n$  (modulo
word size) represents IRQ $n$, the bit will be set if the
corresponding IRQ was raised. This supports the use of a single handler for
multiple IRQs.

\obj{IRQHandler} capabilities represent the ability of a thread to
configure a certain interrupt. They have three methods:

\begin{description}
    \item[\apifunc{seL4\_IRQHandler\_SetNotification}{irq_handlersetnotification}]
    specifies the \obj{Notification} the kernel should
    \apifunc{signal}{sel4_signal} when an interrupt occurs. A driver
    may then call \apifunc{seL4\_Wait}{sel4_wait} or \apifunc{seL4\_Poll}{sel4_poll}
    on this notification to
    wait for interrupts to arrive.

    \item[\apifunc{seL4\_IRQHandler\_Ack}{irq_handleracknowledge}]
    informs the kernel that the userspace driver has finished processing
    the interrupt and the microkernel can send further pending or new
    interrupts to the application.

    \item[\apifunc{seL4\_IRQHandler\_Clear}{irq_handlerclear}]
    de-registers the \obj{Notification} from the \obj{IRQHandler} object.
\end{description}

When the system first starts, no \obj{IRQHandler} capabilities are
present. Instead, the initial thread's CSpace contains a single
\obj{IRQControl} capability. This capability may be used to produce
a single \obj{IRQHandler} capability for each interrupt available in the
system. Typically, the initial thread of a system will determine which
IRQs are required by other components in the system, produce an
\obj{IRQHandler} capability for each interrupt, and then delegate the
resulting capabilities as appropriate. Methods on \obj{IRQControl} can
be used for creating \obj{IRQHandler} capabilities for interrupt sources.

\ifxeightsix
\section{IA-32-Specific I/O}

\subsection{Interrupts}
\label{sec:ia32_interrupts}

In addition to managing \obj{IRQHandler} capabilities, IA-32 platforms require
the delivery location in the CPU vectors to be configured. Regardless of where
an interrupt comes from (IOAPIC, MSI, etc) it must be assigned a unique vector
for delivery, ranging from VECTOR\_MIN to VECTOR\_MAX. The rights to allocate
a vector are effectively given through the \obj{IRQControl} capability and can
be considered as the kernel outsourcing the allocation of this namespace to
user level.

\begin{description}
    \item[\apifunc{seL4\_IRQControl\_GetIOAPIC}{ia32_irq_handler_getioapic}] creates
    an \obj{IRQHandler} capability for an IOAPIC interrupt

    \item[\apifunc{seL4\_IRQControl\_GetMSI}{ia32_irq_handler_getmsi}] creates
    an \obj{IRQHandler} capability for an MSI interrupt
\end{description}

\subsection{I/O Ports}
\label{sec:ioports}

On IA-32 platforms, seL4 provides access to I/O ports to user-level threads.
Access to I/O ports is controlled by \obj{IO Port} capabilities. Each
\obj{IO Port} capability identifies a range of ports that can be accessed with
it. Reading from I/O ports is accomplished with the
\apifunc{seL4\_IA32\_IOPort\_In8}{ia32_IO_portin8},
\apifunc{seL4\_IA32\_IOPort\_In16}{ia32_IO_portin16}, and
\apifunc{seL4\_IA32\_IOPort\_In32}{ia32_IO_portin32} methods, which
allow for reading of 8-, 16- and 32-bit quantities.
Similarly, writing to I/O ports is accomplished with the
\apifunc{seL4\_IA32\_IOPort\_Out8}{ia32_IO_portout8},
\apifunc{seL4\_IA32\_IOPort\_Out16}{ia32_IO_portout16}, and
\apifunc{seL4\_IA32\_IOPort\_Out32}{ia32_IO_portout32} methods.
Each of these methods takes as arguments an \obj{IO Port} capability
and an unsigned integer~\texttt{port}, which indicates the I/O port to read from
or write to, respectively.
In each case, \texttt{port} must be within the range of I/O ports identified
by the given \obj{IO Port} capability in order for the method to succeed.

At system initialisation, the initial thread's \obj{CSpace} contains the
master \obj{IO Port} capability, which allows access to all I/O ports.
Other \obj{IO Port} capabilities, which authorise access to a specific
range of I/O Ports, may be derived from this master capability using
the \apifunc{seL4\_CNode\_Mint}{cnode_mint} method. 
The range of I/O ports that the newly created capability should identify
are specified via the 32-bit \texttt{badge} argument provided to 
\apifunc{seL4\_CNode\_Mint}{cnode_mint}. The first port number in the range
occupies the top 16 bits of \texttt{badge}, while the last port number in
the range occupies the bottom 16 bits. The range is interpreted as being
inclusive of these two numbers.

The I/O port methods return error codes upon failure.
A \texttt{seL4\_IllegalOperation} code is returned if port access is
attempted outside the range allowed by the \obj{IO Port} capability. 
Since invocations that
read from I/O ports are required to return two values -- the value read
and the error code -- a structure containing two members, \texttt{result}
and \texttt{error}, is returned from these API calls.

\subsection{I/O Space}
\label{sec:iospace}

I/O devices capable of DMA present a security risk because the CPU's MMU
is bypassed when the device accesses memory. In seL4, device drivers run
in user space to keep them out of the trusted computing base.
A malicious or buggy device driver may, however, program the device to
access or corrupt memory that is not part of its address space, thus
subverting security. To mitigate this threat, seL4 provides support for
the IOMMU on Intel IA-32-based platforms. An IOMMU allows memory to be
remapped from the device's point of view. It acts as an MMU for the
device, restricting the regions of system memory that it can access.
More information can be obtained from Intel's IOMMU documentation \cite{extra:vtd}.

seL4-based systems that wish to utilise DMA must have an IOMMU.
This restriction results from the fact that seL4 provides no way to obtain the physical address of a \obj{Page} from its capability.
Hence, applications are unable to accurately instruct devices,
at which address they should directly address the physical memory.
Instead, frames of memory must be mapped into the device's address space using seL4's IOMMU primitives.

Two new objects are provided by the kernel to abstract the IOMMU:
\begin{description}

    \item[\obj{IOSpace}] This object represents the address space associated
    with a hardware device on the PCI bus. It represents the right to
    modify a device's memory mappings.

    \item[\obj{IOPageTable}] This object represents a node in the multilevel
    page-table structure used by IOMMU hardware to translate hardware
    memory accesses.

\end{description}

\obj{Page} capabilities are used to represent the actual frames that are
mapped into the I/O address space. A \obj{Page} can be mapped into
either a \obj{VSpace} or an \obj{IOSpace} but never into both at the same time.

\obj{IOSpace} and \obj{VSpace} fault handling differ significantly.
\obj{VSpace} page faults are redirected to the thread's exception handler (see \autoref{sec:faults}), 
which can take the
appropriate action and restart the thread at the faulting instruction.
There is no concept of an exception handler for an \obj{IOSpace}. Instead, faulting
transactions are simply
aborted; the device driver must correct the cause of the fault and retry
the DMA transaction.

An initial master \obj{IOSpace} capability is provided in the initial thread's
CSpace. An \obj{IOSpace} capability for a specific device is created by
using the \apifunc{seL4\_CNode\_Mint}{cnode_mint} method, passing the
PCI identifier of the device as the low 16 bits of the \texttt{badge} argument, and
a Domain ID as the high 16 bits of the \texttt{badge} argument.
PCI identifiers are explained fully in the PCI specification 
\cite{Shanley:PCISA}, but are briefly described here. A PCI identifier is
a 16-bit quantity.  The first 8 bits identify the bus that the device is on.
The next 5 bits are the device identifier: the number of the device on
the bus. The last 3 bits are the function number. A single device may
consist of several independent functions, each of which may be addressed
by the PCI identifier.
Domain IDs are explained fully in the Intel IOMMU documentation \cite{extra:vtd}.
There is presently no way to query seL4 for how many Domain IDs are supported by
the IOMMU and the \apifunc{seL4\_CNode\_Mint}{cnode_mint} method will fail if an
unsupported value is chosen.

The IOMMU page-table structure has three levels.
Page tables are mapped into an \obj{IOSpace} using the \apifunc{seL4\_IA32\_IOPageTable\_Map}{ia32_io_pagetable_map} method.
This method takes the \obj{IOPageTable} to map, the \obj{IOSpace} to map into 
and the address to map at. Three levels of page tables must be mapped before
a frame can be mapped successfully. A frame is mapped with the
\apifunc{seL4\_IA32\_Page\_MapIO}{ia32_page_mapio} method whose parameters are analogous to
the corresponding method that maps \obj{Page}s into \obj{VSpaces} (see \autoref{ch:vspace}), 
namely \apifunc{seL4\_IA32\_Page\_Map}{ia32_page_map}.

Unmapping is accomplished with the usual unmap (see \autoref{ch:vspace}) API 
call,
\apifunc{seL4\_IA32\_Page\_Unmap}{ia32_page_unmap}.

More information about seL4's IOMMU abstractions can be found in \cite{Palande:M}.
\fi
