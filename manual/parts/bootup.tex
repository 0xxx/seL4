%
% Copyright 2014, General Dynamics C4 Systems
%
% This software may be distributed and modified according to the terms of
% the GNU General Public License version 2. Note that NO WARRANTY is provided.
% See "LICENSE_GPLv2.txt" for details.
%
% @TAG(GD_GPL)
%

\chapter{\label{ch:bootup}System Bootstrapping}

\section{Initial Thread's Environment}

The seL4 kernel creates a minimal boot environment for the initial thread.
This environment consists of the initial thread's TCB, CSpace and VSpace,
consisting of frames that contain the userland image (code/data of the initial
thread) and the IPC buffer.
The initial thread's CSpace consists of exactly one CNode
which contains capabilities to the initial
thread's own resources was well as to all available global resources.
The CNode size can be configured at compile time (default is $2^{12}$
slots), but the guard is always chosen so that the CNode resolves exactly
32 bits. This means, the first slot of the CNode has CPTR 0x0, the
second slot has CPTR 0x1 etc.

The first 12 slots contain specific capabilities as listed in
\autoref{tab:cnode_content}.

\begin{table}[htb]
  \begin{center}
    \caption{Initial thread's CNode content.}
    \label{tab:cnode_content}
    \begin{tabularx}{\textwidth}{llX}
      \toprule
      CPTR & Enum Constant & Capability \\
      \midrule
      0x0 & \texttt{seL4\_CapNull}                & null \\
      0x1 & \texttt{seL4\_CapInitThreadTCB}       & initial thread's TCB \\
      0x2 & \texttt{seL4\_CapInitThreadCNode}     & initial thread's CNode \\
      0x3 & \texttt{seL4\_CapInitThreadPD}        & initial thread's page directory \\
      0x4 & \texttt{seL4\_CapIRQControl}          & global IRQ controller (see \autoref{sec:interrupts}) \\
      0x5 & \texttt{seL4\_CapASIDControl}         & global ASID controller (see \autoref{ch:vspace}) \\
      0x6 & \texttt{seL4\_CapInitThreadASIDPool}  & initial thread's ASID pool (see \autoref{ch:vspace}) \\
      0x7 & \texttt{seL4\_CapIOPort}              & global I/O port cap, null cap if unsupported
\ifxeightsix
(see \autoref{sec:ioports})
\fi
\\
      0x8 & \texttt{seL4\_CapIOSpace}             & global I/O space cap, null cap if unsupported
\ifxeightsix
(see \autoref{sec:iospace})
\fi
\\
      0x9 & \texttt{seL4\_CapBootInfoFrame}       & BootInfo frame (see \autoref{ch:bootup:bootinfo}) \\
      0xa & \texttt{seL4\_CapInitThreadIPCBuffer} & initial thread's IPC buffer (see \autoref{sec:messageinfo}) \\
      0xb & \texttt{seL4\_CapDomain}              & domain cap (see \autoref{sec:domains}) \\
      \bottomrule
    \end{tabularx}
  \end{center}
\end{table}

\section{\label{ch:bootup:bootinfo}BootInfo Frame}

CNode slots with CPTR \texttt{seL4\_NumInitialCaps} (defined in the seL4
userland library) and above are filled dynamically during
bootstrapping. Their exact contents depend on the userland image size,
platform configuration (devices) etc. In order to tell the initial thread
which capabilities are stored where in its CNode, the kernel provides
a \emph{BootInfo Frame} which is mapped into the initial thread's address
space. The mapped address is chosen by the kernel and given to the initial
thread via a CPU register. The initial thread can access this address
through the function \texttt{seL4\_GetBootInfo()}.

The BootInfo Frame contains the C struct described in
\autoref{tab:bootinfo_struct}.
It is defined in the seL4 userland library. Besides talking about
capabilities, it also informs the initial thread about
the current platform's configuration.

The type \texttt{seL4\_SlotRegion} is a C struct
which contains \texttt{start} and \texttt{end} slot CPTRs. It denotes a region
of slots in the initial thread's CNode, starting with CPTR \texttt{start} and with
\texttt{end} being the CPTR of the first slot after the region ends, i.e.\
\texttt{end - 1} points to the last slot of the region.

\begin{table}[htb]
  \begin{center}
    \caption{BootInfo struct.}
    \label{tab:bootinfo_struct}
    \begin{tabularx}{\textwidth}{llX}
      \toprule
      Field Type & Field Name & Description \\
      \midrule
      \texttt{seL4\_Word}           & \texttt{nodeID}                  & node ID (see \autoref{ch:bootup:multikernel}) \\
      \texttt{seL4\_Word}           & \texttt{numNodes}                & number of nodes (see \autoref{ch:bootup:multikernel}) \\
      \texttt{seL4\_Word}           & \texttt{numIOPTLevels}           & number of I/O page-table levels (0 if no IOMMU) \\
      \texttt{seL4\_IPCBuffer*}     & \texttt{ipcBuffer}               & pointer to the initial thread's IPC buffer \\
      \texttt{seL4\_SlotRegion}     & \texttt{empty}                   & empty slots (null caps) \\
      \texttt{seL4\_SlotRegion}     & \texttt{sharedFrames}            & see \autoref{ch:bootup:multikernel} \\
      \texttt{seL4\_SlotRegion}     & \texttt{userImageFrames}         & frames containing the userland image \\
      \texttt{seL4\_SlotRegion}     & \texttt{userImagePaging}         & userland-image paging structure caps \\
      \texttt{seL4\_SlotRegion}     & \texttt{untyped}                 & untyped-memory capabilities \\
      \texttt{seL4\_Word[]}         & \texttt{untypedPaddrList}        & array of untyped-memory physical addresses \\
      \texttt{uint8\_t[]}           & \texttt{untypedSizeBitsList}     & array of untyped-memory sizes ($2^n$ bytes) \\
      \texttt{uint8\_t}             & \texttt{initThreadCNodeSizeBits} & CNode size ($2^n$ slots) \\
      \texttt{seL4\_Word}           & \texttt{numDeviceRegions}        & number of device memory regions \\
      \texttt{seL4\_DeviceRegion[]} & \texttt{deviceRegions}           & device memory regions (see \autoref{tab:device_region_struct}) \\
      \texttt{uint32\_t}            & \texttt{initThreadDomain}        & domain of the initial thread (see \autoref{sec:domains}) \\
      \bottomrule
    \end{tabularx}
  \end{center}
\end{table}

The capabilities in \texttt{userImageFrames} are
ordered such that the first capability references the first frame of the
userland image and so on.
The capabilities in \texttt{userImagePaging} are ordered in descending order
of paging structure size. Within a given paging structure size, capabilities are
ordered by the virtual address at which the corresponding objects are mapped
into the initial thread's address space.

Userland always knows to which virtual addresses its own
code and data is mapped (e.g.\ in GCC, with the standard linker script, the
symbols \texttt{\_\_executable\_start} and \texttt{\_end} are available).
Userland can therefore
infer the virtual address behind each userland frame and paging structure cap.

Untyped memory is given in no particular order. The array entry
\texttt{untypedSizeBitsList[i]} stores the untyped-memory size ($2^n$ bytes) of
the i-th untyped cap of the slot region \texttt{untyped}. Therefore, the array
length is at least \texttt{untyped.end - untyped.start}. The actual length is
hardcoded in the kernel and irrelevant to the reader of the array.
The same is true for the array \texttt{untypedPaddrList}. For each
untyped-memory capability, it stores the physical addresses backing the
untyped memory. This allows
userland to infer physical memory addresses of retyped frames and use them to
initiate DMA transfers when no IOMMU is available. The kernel makes no
guarantees about certain sizes of untyped memory being available except that it
provides a compile-time-configurable minimum number of 4K untyped capabilities
(default is 12).

\begin{table}[htb]
  \begin{center}
    \caption{DeviceRegion struct.}
    \label{tab:device_region_struct}
    \begin{tabular}{lll}
      \toprule
      Field Type & Field Name & Description \\
      \midrule
      \texttt{seL4\_Word}       & \texttt{basePaddr}     & physical base address of the device region \\
      \texttt{seL4\_Word}       & \texttt{frameSizeBits} & size ($2^n$ bytes) of the frames used \\
      \texttt{seL4\_SlotRegion} & \texttt{frames}        & capabilities to the frames covering the region \\
      \bottomrule
    \end{tabular}
  \end{center}
\end{table}

The kernel creates frames covering each physical memory region associated with
a memory-mapped device. These device regions are either hardcoded (e.g.\ on
embedded platforms) or discovered at boot time by the kernel through a PCI bus
scan. The physical base address of each region is stored in \texttt{basePaddr}.
The slot region \texttt{frames} identifies all frame caps used to back this
region. They are ordered, so the first frame of the region is referenced by the
first cap in this slot region and is backed by the physical address
\texttt{basePaddr}. All frames have the same size: $2^{frameSizeBits}$ bytes.
Hence, the size of the whole region is:
\texttt{(frames.end - frames.start) << frameSizeBits}

The array \texttt{deviceRegions} of the BootInfo struct stores all available
device regions (i.e.\ their structs). There are \texttt{numDeviceRegions} of
them available in the array.

If the platform has an seL4-supported IOMMU, \texttt{numIOPTLevels} contains
the number of IOMMU-page-table levels. This information is needed by userland
when constructing an IOMMU address space (IOSpace). If there is no IOMMU
support, \texttt{numIOPTLevels} is \texttt{0}.

\ifxeightsix
\section{Boot Command-line Arguments}

On IA-32, seL4 accepts boot command-line arguments which are passed to the
kernel via a multiboot-compliant bootloader (e.g.\ GRUB, syslinux). Multiple
arguments are separated from each other by whitespace. Two forms of arguments
are accepted:
(1) key-value arguments of the form ``key=value'' and (2) single keys of the
form ``key''. The value field of the key-value form may be a string, a decimal
integer, a hexadecimal integer beginning with ``0x'', or an integer list where
list elements are separated by commas.
Keys and values can't have any whitespace in them and there can be no
whitespace before or after an ``='' or a comma either.
Arguments are listed in \autoref{tab:bootargs} along with their default values (if left unspecified).


\begin{table}[htb]
    \caption{IA-32 boot command-line arguments.}
        \begin{tabularx}{\textwidth}{lXX}
            \toprule
              Key & Value & Default \\
            \midrule
            \texttt{console\_port} &
            I/O-port base of the serial port that the kernel prints to
            (if compiled in debug mode) &
            0x3f8 \\
            \texttt{debug\_port} &
            I/O-port base of the serial port that is used for kernel debugging
            (if compiled in debug mode) &
            0x3f8 \\
            \texttt{disable\_iommu} &
            none &
            The IOMMU is enabled by default on VT-d-capable platforms \\
            \texttt{max\_num\_nodes} &
            Maximum number of seL4 nodes that can be started up (see \autoref{ch:bootup:multikernel}) &
            1 \\
            \texttt{num\_sh\_frames} &
            Number of frames shared between seL4 nodes (see \autoref{ch:bootup:multikernel}) &
            0 \\
            \bottomrule
        \end{tabularx}
    \label{tab:bootargs}
\end{table}
\fi

\section{\label{ch:bootup:multikernel}Multikernel Bootstrapping}

On ARM, seL4 is uniprocessor only and does not support multikernel
\cite{Baumann_BDHIPRSS_09} bootstrapping. Therefore, the field \texttt{nodeID}
of the BootInfo struct
will always be \texttt{0}, \texttt{numNodes} will be \texttt{1} and
\texttt{sharedFrames} will be an empty region.

\ifxeightsix
On IA-32, seL4 can be bootstrapped as a multikernel
by setting the boot command-line argument \texttt{max\_num\_nodes} to a value
\texttt{>1}.
Each available CPU core will then run one isolated \emph{node} of seL4, up to
the maximum number specified.
The available physical memory is partitioned equally between nodes.
All device frames are given to all nodes. The nodes' initial threads have to
coordinate access to these device frames, e.g.\ by defining which node is
responsible for which device.
IOMMU management can only be performed by the first node, the other nodes
are not given a global IOSpace capability. 
There is also a hard upper limit of number of nodes defined at compile time
(default is 8, but it can be increased to 256).

Nodes are isolated from each other, except for
\emph{shared frames}. When bootstrapping, the kernel creates a number of 4K
userland frames which are shared between nodes. This number can be configured
via the boot command-line argument \texttt{num\_sh\_frames}. The shared frames
will appear in the CNode of each node's initial thread. They are given to
each initial thread in the same order. In the BootInfo struct, the field
\texttt{sharedFrames} contains their slot region.

Shared frames can be used by userland to implement
shared data structures,
message passing and synchronisation mechanisms. Individual frames can, for
example, be minted and handed out to subsystems. This allows connecting them
across nodes in a fine-grained manner. Each node has a node ID (field
\texttt{nodeID} in BootInfo) whereas the number of nodes can be obtained from
the field \texttt{numNodes}.

Userland images are given to the kernel by a multiboot-compliant bootloader
(e.g.\ GRUB, syslinux) via \emph{boot modules}. Each boot module contains an
ELF file of a userland image. If there is only one userland image,
each node will get its own copy of that userland image. If
multiple userland images are given, the first node gets the first image, the
second node the second image etc. If there are more nodes than images, each
remaining node gets a copy of the last image.

If the kernel is compiled in debug mode, each node can be assigned a separate
console port and debug port (see \autoref{tab:bootargs}) which is done by
specifying an array of I/O-port base addresses. For example,
\texttt{console\_port=0x3f8,0x2f8,0x3e8,0x2e8}
assigns port 0x3f8 to node 0, port 0x2f8 to node 1, etc. The remaining nodes
have no assigned port and produce no console output. The argument
\texttt{debug\_port} works in exactly the same way.

Further details can be obtained from \cite{vonTessin_10} which mainly talks
about the formal aspects of seL4's multikernel version but also contains
details about the bootstrapping. More recent additional information can be
found in \cite{vonTessin_12}, which focusses less on the formal side
and more on the OS side.
\fi
