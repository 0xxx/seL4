%
% Copyright 2014, General Dynamics C4 Systems
%
% This software may be distributed and modified according to the terms of
% the GNU General Public License version 2. Note that NO WARRANTY is provided.
% See "LICENSE_GPLv2.txt" for details.
%
% @TAG(GD_GPL)
%

\chapter{\label{ch:threads}Threads and Execution}

\section{Threads \& Scheduling Contexts}
\label{sec:threads}

seL4 provides threads to represent an execution context, while scheduling contexts are used to manage
processor time.
A thread is represented in seL4 by its thread control block
object (\obj{TCB}) and a scheduling context by a scheduling context object.
Threads cannot run unless they are bound to, or receive a scheduling context.

\subsection{Thread control blocks}

Each \obj{TCB} has an associated CSpace (see
\autoref{ch:cspace}) and VSpace (see \autoref{ch:vspace}) which
may be shared with other threads. A \obj{TCB} may also have an IPC buffer
(see  \autoref{ch:ipc}), which is used to pass extra arguments during IPC
or kernel object invocation that do not fit in the architecture-defined message
registers. While it is not compulsory that a thread has an IPC buffer,
it will not be able to perform most kernel invocations, as they require
cap transfer.

%FIXME: there is much more information held in the TCB!

\subsection{Thread Creation}

Like other objects, \obj{TCB}s are created with the
\apifunc{seL4\_Untyped\_Retype}{untyped_retype} method (see
\autoref{sec:kernmemalloc}). A newly created thread is initially inactive. It
is configured by setting its CSpace and VSpace with the
\apifunc{seL4\_TCB\_SetSpace}{tcb_setspace}
or \apifunc{seL4\_TCB\_Configure}{tcb_configure} methods and then calling
\apifunc{seL4\_TCB\_WriteRegisters}{tcb_writeregisters} with an initial stack pointer and instruction
pointer. The thread can then be activated either by setting the
\texttt{resume\_target} parameter in the \apifunc{seL4\_TCB\_WriteRegisters}{tcb_writeregisters} invocation to true
or by seperately calling the \apifunc{seL4\_TCB\_Resume}{tcb_resume} method.

\subsection{Thread Deactivation}
\label{sec:thread_deactivation}

The \apifunc{seL4\_TCB\_Suspend}{tcb_suspend} method deactivates a thread.
Suspended threads can later be resumed.
Their suspended state can be retrieved with the
\apifunc{seL4\_TCB\_ReadRegisters}{tcb_readregisters} and
\apifunc{seL4\_TCB\_CopyRegisters}{tcb_copyregisters} methods.
They can also be reconfigured and
reused or left suspended indefinitely if not needed. Threads will be
automatically suspended when the last capability to their \obj{TCB} is
deleted.
When threads are suspended, any active IPCs or signals are cancelled.
% an example of which is demonstrated in \nameref{ex:second_thread}.

\subsection{Scheduling Contexts}
\label{sec:scheduling_contexts}

Access to CPU execution time is controlled through scheduling context objects.
Scheduling contexts consist of a tuple of
\textit{budget (b)} and \textit{period (p)}, both in microseconds, set by \apifunc{seL4\_SchedControl\_Configure}{schedcontrol_configure} (see \autoref{sec:sc_creation}).
The tuple $(b, p)$ forms an upper bound on the thread's execution --
the kernel will not permit a thread to run for more than $b$ out of every $p$ microseconds.
However, $\frac{b}{p}$ does not represent a lower bound on execution, as a thread must have the highest or equal highest priority of all runnable threads to be guaranteed to be scheduled at all.

A scheduling context that has budget available is referred to as \emph{active}.
Whenever a thread is executing it consumes the budget from its current scheduling context.
Once the budget is exhausted the thread is preempted and will not be schedulable again until the period has passed, at which
point the budget will be replenished.
When a thread's budget is exhausted, the next runnable thread at that priority with an active scheduling context will be chosen by the scheduler.
When $b = p$, $b$ simply acts as a timeslice for a thread, as the budget is always replenished immediately after it expires, however the thread will be preempted.

The system call \apifunc{seL4\_SchedContext\_Yield}{schedcontext_yield} can be used to sacrifice any remaining budget and block until the budget is replenished.
This allows for periodic tasks to be implemented by setting their scheduling context's period and having them call \apifunc{seL4\_SchedContext\_Yield}{schedcontext_yield} when their job is complete.

Threads can be bound to scheduling contexts using \apifunc{seL4\_TCB\_Configure}{tcb_configure} or
\apifunc{seL4\_SchedContext\_Bind}{schedcontext_bind}, both invocations have the same effect although \apifunc{seL4\_TCB\_Configure}{tcb_configure} allows more thread fields to be set with only one kernel entry.
When a thread is bound to a scheduling context, if it is in a runnable state and the scheduling context is active, it will be added to the scheduler.

Threads can optionally generate exceptions when they attempt to run without available budget, see \autoref{sec:exceptions}.

\subsection{Passive Threads}
\label{sec:passive}

Threads can be unbound from a scheduling context with \apifunc{seL4\_SchedContext\_UnbindObject}{schedcontext_unbindobject}.
This is distinct from suspending a thread, in that threads that are blocked waiting in an endpoint or notification queue will remain
in the queue and can still recieve messages and signals.
However, the unbound thread will not be schedulable again until it receives a scheduling context.
Threads without scheduling contexts are referred to as \emph{passive} threads, as they cannot execute without the action of another thread.

\subsection{Scheduling Context Creation}
\label{sec:sc_creation}

Like other objects, scheduling contexts are created from untyped memory using \apifunc{seL4\_UntypedRetype}{untyped_retype}.
On creation, scheduling contexts are empty, representing 0\% of CPU execution time.
To populate a scheduling context with parameters, one must invoke the \obj{SchedControl} capability, which provides access to CPU time management and is provided to the initial task at run time.
Scheduling context parameters can then be set and updated using \apifunc{seL4\_SchedControl\_Configure}{schedcontrol_configure}, which allows the budget and period to be specified.

The kernel does not conduct any schedulability tests, as task admission is left to user-level policy and can be conducted online or offline, statically or dynamically or not at all.

\subsection{Scheduling Context Donation and Borrowing}

In addition to explicitly binding and removing scheduling contexts through \apifunc{seL4\_SchedContext\_Bind}{schedcontext_bind} and \apifunc{seL4\_SchedContext\_UnbindObject}{schedcontext_unbindobject}, scheduling contexts can move between threads over IPC.
Scheduling contexts are donated implicitly when the system calls \apifunc{seL4\_Call}{sel4_call} and \apifunc{seL4\_SignalRecv}{sel4_signalrecv} are used to communicate with a passive thread.
When \apifunc{seL4\_Call}{sel4_call} is used, the generated reply cap ensures that the callee is merely borrowing the scheduling context: when the reply cap is consumed by a reply message being sent the scheduling context will be returned to the caller.
If the reply cap is revoked, and the callee holds the scheduling context, the scheduling context will be returned to the caller.
However, if in a deep call chain and a reply cap in the middle of the call chain is revoked, such that the callee does not possess the scheduling context, the thread will be removed from the call chain and the scheduling context will remain where it is.

Consider an example where thread A calls thread B which calls thread C.
If while C holds the scheduling context, B's reply cap to A is revoked, then the scheduling context will remain with C.
However, a call chain will remain between A and C, such that if C's reply cap is revoked, or invoked, the scheduling context will return to A.

\apifunc{seL4\_SignalRecv}{sel4_signalrecv} only offers scheduling context donation: there is no guarantee that the scheduling context will return.

Scheduling contexts can also be bound to notification objects using \apifunc{seL4\_SchedContext\_Bind}{schedcontext_bind} and unbound using \apifunc{seL4\_SchedContext\_UnbindObject}{schedcontext_unbindobject}.
If a signal is delivered to a notification object with a passive thread blocked waiting on it, the passive thread will receive the scheduling context that is bound to the notification object.
The scheduling context is returned when the thread blocks on the notification object.
This feature allows for passive servers to use notification binding (See \autoref{sec:notification-binding}).

Scheduling contexts can be unbound from all objects (notification objects and TCBs that are bound or have received a scheduling context through donation) using \apifunc{seL4\_SchedContext\_Unbind}{schedcontext_unbind}.

\subsection{Scheduling algorithm}
\label{sec:sched}

seL4 uses a preemptive, tickless, scheduler with 256 priority levels (0 --- 255) and 0 --- 4 criticality levels.
Thread scheduling in seL4 is controlled via two distinct values: priorities and criticalities, which facilitate mixed-criticality scheduling.
The kernel maintains a criticality level and only threads of criticality higher or equal to the current kernel criticality level are eligible for scheduling.
Additionally, threads are only eligible for scheduling if they have an active scheduling context.
Of threads eligible for scheduling, the highest priority thread in a runnable state is chosen.

Thread priority (structure \texttt{seL4\_Prio\_t}) consists of four values as follows:

\begin{description}
    \item[Priority] the priority a thread will be scheduled with.
    \item[Maximum controlled priority (MCP)] the highest priority a thread can set itself or another thread to.
    \item[Criticality] the criticality of a thread.
    \item[Maximum controlled criticality] the highest criticality a thread can set itself or another thread to.
\end{description}

Threads of sufficient maximum contrlled priority and with possession of the appropriate scheduling context capability can manipulate the scheduler and implement user-level schedulers using \apifunc{seL4\_SchedContext\_YieldTo}{schedcontext_yieldto} and \apifunc{seL4\_SchedContext\_Consumed}{schedcontext_consumed}.

\subsection{Priorities}

Scheduling contexts provide access to and an upper bound on exection CPU time, however when a thread executes is determined by thread priority.

All threads have a maximum controlled priority (MCP) and a priority, the latter being the effective priority of the thread.
When a thread creates or modifies another thread, it can only set the
other thread's priority and MCP to be less than or equal to its own MCP. Thread priority and MCP can be
set with \apifunc{seL4\_TCB\_Configure}{tcb_configure} and
\apifunc{seL4\_TCB\_SetPriority}{tcb_setpriority}, \apifunc{seL4\_TCB\_SetMCPriority}{tcb_setmcpriority} methods.

Consequently, access to CPU is a function of thread MCPs, scheduling contexts and the \obj{SchedControl} capability.
The kernel will enforce that threads do not exceeed the budget in their scheduling context for any given period, and that the highest priority thread will always run, however it is up to the system designer to make sure the entire system is schedulable.

\subsection{Criticalities}
\label{sec:criticality}

Thread criticality provides the mechanism for implementing mixed-criticality scheduling in an efficient way on seL4.
Criticality allows the system to change operating mode, with the understanding that the highest priority thread is not neccessarily the most important thread.
Should a high criticalty thread need more time, low criticality threads can be removed from the scheduler by changing the kernel criticality level with \apifunc{seL4\_SchedControl\_SetCriticality}{schedcontrol_setcriticality}, which is $O(n)$ in the number of threads with criticality greater than or equal to the criticality being set.
Criticality can be restored with the same function.

Thread criticality can be set with \apifunc{seL4\_TCB\_SetCriticality}{tcb_setcriticality}, and like thread priorities, criticality assignment is controlled by the maximum control criticality, set with \apifunc{seL4\_TCB\_SetMCCriticality}{tcb_setmcc}.
Both fields can be set with \apifunc{seL4\_TCB\_Configure}{tcb_configure}.

\subsection{Exceptions}
\label{sec:exceptions}

Each thread has two associated exception-handler endpoints, a \emph{standard} exception handler and a \emph{timeout} exception handler.
If the thread
causes an exception, the kernel creates an IPC message with the relevant
details and sends the appropriate endpoint. This
thread can then take the appropriate action. Fault IPC messages are
described in \autoref{sec:faults}.

Exception-handler
endpoints can be set with the \apifunc{seL4\_TCB\_SetSpace}{tcb_setspace} or
\apifunc{seL4\_TCB\_Configure}{tcb_configure} methods.
With these methods, a capability address for the exception handler can be associated with a thread.
This address is then used to lookup the handler endpoint, and the capability to the endpoint is installed into the threads' kernel CNode.
For threads without an exception handler, a null capability can be used, however the consequences of are different per exception handler type.

Before raising an exception the handler capability is validated - the kernel does not perform another lookup, but checks that the capability is an endpoint with the correct rights.

The exception endpoint must have send and grant rights. Replying to the
exception message restarts the thread. For certain exception types, the contents of
the reply message may be used to set the values in the registers of the
thread being restarted.
See \autoref{sec:faults} for details.

\subsubsection{Standard Exceptions}

The standard exception handler is used when a fault is triggered by a thread which cannot be recovered from without action by another thread.
For example, if a thread raises a fault due to an unmapped virtual memory page, the thread cannot make any more progress until the page is mapped.
If a thread experiences a fault that would trigger the standard exception handler while it is set to a null capability, the kernel will pause the thread and it will not run again.
This is because without action by another thread, standard exceptions cannot be recovered from.
Consequently threads without standard exception handlers should be trusted not to fault at all.

Standard exception handlers can be passive, in which case they will run on the scheduling context of the faulting thread.

\subsubsection{Timeout Exceptions}
\label{sec:timeout-exceptions}

Timeout faults are raised when a thread attempts to run but has no available budget, and if that thread has a valid timeout exception handler capability.
The handling of timeout faults is not compulsory: if a thread does not have a timeout fault handler, a fault will not be raised and the thread will continue running when it's budget is replenished.
This allows temporally sensitive threads to handle budget overruns while other threads may ignore them.

Timeout faults are registered per thread, which means that while clients may not have a timeout fault handler, servers may, allowing single-threaded, time-sensitive, passive servers to use a timeout exception handler to recover from malicious or untrusted clients whose budget expires while the server is completing the request.
Timeout faults handlers can use \apifunc{seL4\_CNode\_SwapTCBCaller}{cnode_swaptcbcaller} to save the servers reply capability and reply with an error to the client, then resetting the server to handle the next client request.

If a reply message is sent to a nested server and a scheduling context without available budget returned, another timeout fault will be generated if the nested server also has a timeout fault handler.

Additionally, if the system criticality is changed while a thread with higher criticality than the system criticality is running on a scheduling context that is bound to a thread with criticality lower than the system criticality, a timeout exception will be raised.

\subsection{Message Layout of the Read-/Write-Registers Methods}
\label{sec:read_write_registers}

The registers of a thread can be read and written with the
\apifunc{seL4\_TCB\_ReadRegisters}{tcb_readregisters} and \apifunc{seL4\_TCB\_WriteRegisters}{tcb_writeregisters} methods. The register contents are transferred via the IPC buffer. The IPC buffer locations that registers are copied to/from are given below.

\ifxeightsix
\subsubsection{IA-32}

\begin{tabularx}{\textwidth}{p{0.4\textwidth}X}
\toprule
\textbf{Register} & \textbf{IPC Buffer location} \\
\midrule
\reg{EIP} & \ipcbloc{IPCBuffer[0]} \\
\reg{ESP} & \ipcbloc{IPCBuffer[1]} \\
\reg{EFLAGS} & \ipcbloc{IPCBuffer[2]} \\
\reg{EAX} & \ipcbloc{IPCBuffer[3]} \\
\reg{EBX} & \ipcbloc{IPCBuffer[4]} \\
\reg{ECX} & \ipcbloc{IPCBuffer[5]} \\
\reg{EDX} & \ipcbloc{IPCBuffer[6]} \\
\reg{ESI} & \ipcbloc{IPCBuffer[7]} \\
\reg{EDI} & \ipcbloc{IPCBuffer[8]} \\
\reg{EBP} & \ipcbloc{IPCBuffer[9]} \\
\reg{TLS\_BASE} & \ipcbloc{IPCBuffer[10]} \\
\reg{FS} & \ipcbloc{IPCBuffer[11]} \\
\reg{GS} & \ipcbloc{IPCBuffer[12]} \\
\bottomrule
\end{tabularx}
\fi

\subsubsection{ARM}

\begin{tabularx}{\textwidth}{p{0.4\textwidth}X}
\toprule
\textbf{Register} & \textbf{IPC Buffer location} \\
\midrule
\reg{PC} & \ipcbloc{IPCBuffer[0]} \\
\reg{SP} & \ipcbloc{IPCBuffer[1]} \\
\reg{CPSR} & \ipcbloc{IPCBuffer[2]} \\
\reg{R0-R1} & \ipcbloc{IPCBuffer[3-4]} \\
\reg{R8-R12} & \ipcbloc{IPCBuffer[5-9]} \\
\reg{R2-R7} & \ipcbloc{IPCBuffer[10-15]} \\
\reg{R14} & \ipcbloc{IPCBuffer[16]} \\
\bottomrule
\end{tabularx}


\section{Faults}
\label{sec:faults}

A thread's actions may result in a fault. Faults are delivered to the
thread's exception handler so that it can take the appropriate action.
The fault type is specified in the message label and is one of:
seL4\_Fault\_CapFault, seL4\_Fault\_VMFault, seL4\_Fault\_UnknownSyscall, seL4\_Fault\_UserException,
seL4\_Fault\_TimeoutFault or seL4\_Fault\_NullFault (indicating no fault occured and this is a normal
IPC message).

\subsection{Capability Faults}

Capability faults may occur in two places. Firstly, a capability fault
can occur when lookup of a capability referenced by a
\apifunc{seL4\_Call}{sel4_call} or \apifunc{seL4\_Send}{sel4_send} system call
failed (\apifunc{seL4\_NBSend}{sel4_nbsend} calls on
invalid capabilities silently fail). In this case, the capability
on which the fault occurred may be the capability being invoked or an
extra capability passed in the \texttt{caps} field in the IPC buffer.

Secondly, a capability fault can occur when \apifunc{seL4\_Recv}{sel4_recv} or \apifunc{seL4\_NBRecv}{sel4_nbrecv}
is called on a capability that does not exist, is not an endpoint or notification capability or does not have
receive permissions.

Replying to the fault IPC will restart the faulting thread. The contents of the
IPC message are given in \autoref{tbl:ipc_contents}.\\

\begin{table}[htb]
\noindent\begin{tabularx}{\textwidth}{XX}
\toprule
    \textbf{Meaning} & \textbf{ IPC buffer location} \\
\midrule
    Address at which to restart execution & \ipcbloc{seL4\_CapFault\_IP} \\
    Capability address & \ipcbloc{seL4\_CapFault\_Addr} \\
In receive phase (1 if the fault happened during a receive system call, 0
    otherwise) & \ipcbloc{seL4\_CapFault\_InRecvPhase} \\
Lookup failure description. As described in \autoref{sec:lookup_fail_desc} &
    \ipcbloc{seL4\_CapFault\_LookupFailureType} \\
\bottomrule
\end{tabularx}
\caption{\label{tbl:ipc_contents}Contents of an IPC message.}
\end{table}

\subsection{Unknown Syscall}
\label{sec:unknown-syscall}

This fault occurs when a thread executes a system call with a syscall
number that is unknown to seL4.
The register set
of the faulting thread is passed to the thread's exception handler so that it
may, for example, emulate the system call if a thread is being
virtualised.

Replying to the fault IPC allows the thread to be restarted
and/or the thread's register set to be modified. If the reply has
a label of zero, the thread will be restarted. Additionally, if the
message length is non-zero, the faulting thread's register set will be
updated as shown in \autoref{tbl:unknown_syscall_result_arm} \ifxeightsix and
\autoref{tbl:unknown_syscall_result_ia32}\fi. In this case, the number of
registers updated is controlled with the length field of the message
tag.

\subsubsection{ARM}

\begin{table}[htb]
\begin{tabularx}{\textwidth}{XXp{0.4\textwidth}}
\toprule
\textbf{Value sent} & \textbf{Register set by reply} & \textbf{IPC buffer location} \\
\midrule
\reg{R0-R7} & (same) & \ipcbloc{seL4\_UnknownSyscall\_R[0-7]} \\
\reg{FaultInstruction} & (same) & \ipcbloc{seL4\_UnknownSyscall\_FaultIP} \\
\reg{SP} & (same) & \ipcbloc{seL4\_UnknownSyscall\_SP} \\
\reg{LR} & (same) & \ipcbloc{seL4\_UnknownSyscall\_LR} \\
\reg{CPSR} & (same) & \ipcbloc{seL4\_UknownSyscall\_CPSR} \\
Syscall number & --- & \ipcbloc{seL4\_UnknownSyscall\_Syscall} \\
\bottomrule
\end{tabularx}
\caption{\label{tbl:unknown_syscall_result_arm}Unknown system call outcome on
the ARM architecture.}
\end{table}

\ifxeightsix
\subsubsection{IA-32}
% FIXME: This table now reflows onto the following page with the paragraph after
% inserted here :(
\begin{table}[htb]
\begin{tabularx}{\textwidth}{XXp{0.4\textwidth}}
\toprule
\textbf{Value sent} & \textbf{Reply register} & \textbf{IPC buffer location} \\
\midrule
\reg{EAX} & (same) & \ipcbloc{seL4\_UnknownSyscall\_EAX} \\
\reg{EBX} & (same) & \ipcbloc{seL4\_UnknownSyscall\_EBX} \\
\reg{ECX} & (same) & \ipcbloc{seL4\_UnknownSyscall\_ECX} \\
\reg{EDX} & (same) & \ipcbloc{seL4\_UnknownSyscall\_EDX} \\
\reg{ESI} & (same) & \ipcbloc{seL4\_UnknownSyscall\_ESI} \\
\reg{EDI} & (same) & \ipcbloc{seL4\_UnknownSyscall\_EDI} \\
\reg{EBP} & (same) & \ipcbloc{seL4\_UnknownSyscall\_EBP} \\
\reg{EIP} & (same) & \ipcbloc{seL4\_UnknownSyscall\_FaultIP} \\
\reg{ESP} & (same) & \ipcbloc{seL4\_UnknownSyscall\_SP} \\
\reg{EFLAGS} & (same) & \ipcbloc{seL4\_UnknownSyscall\_EFLAGS} \\
Syscall number & --- & \ipcbloc{seL4\_UnknownSyscall\_Syscall} \\
\bottomrule
\end{tabularx}
\caption{\label{tbl:unknown_syscall_result_ia32}Unknown system call outcome on
the IA-32 architecture.}
\end{table}
\fi


\subsection{User Exception}

User exceptions are used to deliver architecture-defined exceptions. For
example, such an exception could occur if a user thread attempted to
divide a number by zero.

Replying to the fault IPC allows the thread to be restarted
and/or the thread's register set to be modified. If the reply has
a label of zero, the thread will be restarted. Additionally, if the
message length is non-zero, the faulting thread's register set will be
updated as shown in \autoref{tbl:user_exception_result_arm} \ifxeightsix and
\autoref{tbl:user_exception_result_ia32}\fi. In this case, the number of
registers updated is controlled with the length field of the message
tag.

\subsubsection{ARM}

\begin{table}[htb]
\begin{tabularx}{\textwidth}{XXp{0.4\textwidth}}
\toprule
\textbf{Value sent} & \textbf{Register set by reply} & \textbf{IPC buffer location} \\
\midrule
\reg{FaultInstruction} & (same) & \ipcbloc{seL4\_UserException\_FaultIP} \\
\reg{SP} & (same) & \ipcbloc{seL4\_UserException\_SP} \\
\reg{CPSR} & (same) & \ipcbloc{seL4\_UserException\_CPSR} \\
Exception number & --- & \ipcbloc{seL4\_UserException\_Number} \\
Exception code & --- & \ipcbloc{seL4\_UserException\_Code} \\
\bottomrule
\end{tabularx}
\caption{\label{tbl:user_exception_result_arm}User exception outcome on the ARM
architecture.}
\end{table}

\ifxeightsix
\subsubsection{IA-32}

\begin{table}[htb]
\begin{tabularx}{\textwidth}{XXp{0.4\textwidth}}
\toprule
\textbf{Value sent} & \textbf{Register set by reply} & \textbf{IPC buffer location} \\
\midrule
\reg{EIP} & (same) & \ipcbloc{seL4\_UserException\_FaultIP} \\
\reg{ESP} & (same) & \ipcbloc{seL4\_UserException\_SP} \\
\reg{EFLAGS} & (same) & \ipcbloc{seL4\_UserException\_EFLAGS} \\
Exception number & --- & \ipcbloc{seL4\_UserException\_Number} \\
Exception code & --- & \ipcbloc{seL4\_UserException\_Code} \\
\bottomrule
\end{tabularx}
\caption{\label{tbl:user_exception_result_ia32}User exception outcome on the
IA-32 architecture.}
\end{table}
\fi

\subsection{VM Fault}
\label{sec:vm-fault}

The thread caused a page fault. Replying to the fault IPC will restart
the thread. The contents of the IPC message are given below.\\

\subsubsection{ARM}

\begin{table}[htb]
\begin{tabularx}{\textwidth}{XXX}
\toprule
\textbf{Meaning} & \textbf{IPC buffer location} \\
\midrule
    Program counter to restart execution at. & \ipcbloc{seL4\_VMFault\_IP} \\
Address that caused the fault. & \ipcbloc{seL4\_VMFault\_SP} \\
    Instruction fault (1 if the fault was caused by an instruction fetch). & \ipcbloc{seL4\_VMFault\_PrefetchFault}  \\
Fault status register (FSR). Contains information about the cause of the fault. Architecture dependent. & \ipcbloc{seL4\_VMFault\_FSR} \\
\bottomrule
\end{tabularx}
\caption{\label{tbl:vm_fault_result_arm} VM Fault outcome on the ARM
architecture.}
\end{table}

\ifxeightsix
\subsubsection{IA-32}

\begin{table}[htb]
\begin{tabularx}{\textwidth}{XXX}
\toprule
\textbf{Meaning} & \textbf{IPC buffer location} \\
\midrule
    Program counter to restart execution at. & \ipcbloc{seL4\_VMFault\_IP} \\
Address that caused the fault. & \ipcbloc{seL4\_VMFault\_SP} \\
    Instruction fault (1 if the fault was caused by an instruction fetch). & \ipcbloc{seL4\_VMFault\_PrefetchFault}  \\
Fault status register (FSR). Contains information about the cause of the fault. Architecture dependent. & \ipcbloc{seL4\_VMFault\_FSR} \\
\bottomrule
\end{tabularx}
\caption{\label{tbl:vm_fault_result_ia32} VM fault outcome on the
IA-32 architecture.}
\end{table}
\fi

\subsection{Timeout Fault}
\label{sec:timeout-fault}

Timeout faults are raised when a thread consumes all of its budget and has a timeout fault handler that is not a null capability.
They allow a timeout exception handler to take some action to restore the thread, and deliver a message containting the schedluing context data word, as well as the amount of time consumed since the last timeout fault occured on this scheduling context, or since \apifunc{seL4\_SchedContext\_YieldTo}{schedcontext_yieldto} or \apifunc{seL4\_SchedContext\_Consumed}{schedcontext_consumed} was last called.
Timeout exception handlers can reply to a temporal fault with the registers set in the same format as in outlined in \autoref{sec:read_write_registers}.

\begin{table}[htb]
\noindent\begin{tabularx}{\textwidth}{XX}
\toprule
\textbf{Meaning} & \textbf{IPC buffer location} \\
\midrule
Data word from the scheduling context object that the thread was running on when the fault occured. &
    \ipcbloc{seL4\_TimeoutFault\_Data} \\
Upper 32-bits of microseconds consumed since last reset & \ipcbloc{seL4\_TimeoutFault\_Consumed} \\
    Lower 32-bits of microseconds consumed since last reset & \ipcbloc{seL4\_TimeoutFault\_Consumed\_LowBits} \\
\bottomrule
\end{tabularx}\\ \\
\caption{\label{tbl:tf_message_32} Timeout
    fault outcome on 32-bit architectures.}
\end{table}

