%
% Copyright 2014, General Dynamics C4 Systems
%
% This software may be distributed and modified according to the terms of
% the GNU General Public License version 2. Note that NO WARRANTY is provided.
% See "LICENSE_GPLv2.txt" for details.
%
% @TAG(GD_GPL)
%

\chapter{\label{ch:threads}Threads and Execution}

\section{Threads}
\label{sec:threads}

seL4 provides threads to represent an execution context and manage
processor time. A thread is represented in seL4 by its thread control block
object (\obj{TCB}). Each \obj{TCB} has an associated CSpace (see
\autoref{ch:cspace}) and VSpace (see \autoref{ch:vspace}) which
may be shared with other threads. A \obj{TCB} may also have an IPC buffer
(see  \autoref{ch:ipc}), which is used to pass extra arguments during IPC
or kernel object invocation that do not fit in the architecture-defined message
registers. While it is not compulsory that a thread has an IPC buffer,
it will not be able to perform most kernel invocations, as they require
cap transfer.
Each thread belongs to exactly one security domain (see 
\autoref{sec:domains}).
%FIXME: there is much more information held in the TCB!

\subsection{Thread Creation}

Like other objects, \obj{TCB}s are created with the
\apifunc{seL4\_Untyped\_Retype}{untyped_retype} method (see
\autoref{sec:kernmemalloc}). A newly created thread is initially inactive. It
is configured by setting its CSpace and VSpace with the
\apifunc{seL4\_TCB\_SetSpace}{tcb_setspace}
or \apifunc{seL4\_TCB\_Configure}{tcb_configure} methods and then calling
\apifunc{seL4\_TCB\_WriteRegisters}{tcb_writeregisters} with an initial stack pointer and instruction
pointer. The thread can then be activated either by setting the
\texttt{resume\_target} parameter in the \apifunc{seL4\_TCB\_WriteRegisters}{tcb_writeregisters} invocation to true
or by seperately calling the \apifunc{seL4\_TCB\_Resume}{tcb_resume} method.

\subsection{Thread Deactivation}
\label{sec:thread_deactivation}

The \apifunc{seL4\_TCB\_Suspend}{tcb_suspend} method deactivates a thread.
Suspended threads can later be resumed.
Their suspended state can be retrieved with the 
\apifunc{seL4\_TCB\_ReadRegisters}{tcb_readregisters} and
\apifunc{seL4\_TCB\_CopyRegisters}{tcb_copyregisters} methods.
They can also be reconfigured and
reused or left suspended indefinitely if not needed. Threads will be
automatically suspended when the last capability to their \obj{TCB} is
deleted.
% an example of which is demonstrated in \nameref{ex:second_thread}.

\subsection{Scheduling}
\label{sec:sched}

seL4 uses a preemptive round-robin scheduler with 256 priority levels (0 --- 255).
Thread priority (structure \texttt{seL4\_Prio\_t}) consists of two values, priority and max-priority.
When a thread creates or modifies another thread, it can only set the
other thread's priority and max-priority to be lower than or equal to its own max-priority. Thread priority and max-priority can be
set with \apifunc{seL4\_TCB\_Configure}{tcb_configure} and
\apifunc{seL4\_TCB\_SetPriority}{tcb_setpriority}, \apifunc{seL4\_TCB\_SetMaxPriority}{tcb\_setmaxpriority} methods.

\subsection{Exceptions}

Each thread has an associated exception-handler endpoint. If the thread
causes an exception, the kernel creates an IPC message with the relevant
details and sends this to the endpoint. This
thread can then take the appropriate action. Fault IPC messages are
described in \autoref{sec:faults}.

In order to enable exception handlers, a capability to the exception-handler
endpoint must exist in the CSpace of the thread that generates the exception.
The exception-handler
endpoint can be set with the \apifunc{seL4\_TCB\_SetSpace}{tcb_setspace} or
\apifunc{seL4\_TCB\_Configure}{tcb_configure} method.
With these methods, a capability address for the exception handler can be associated with a thread.
This address is then used to lookup the handler endpoint when an exception is generated.
Note, however, that these methods make no attempt to check whether an endpoint capability exists at the specified
address in the CSpace of the thread. The capability is only looked up
when an exception actually happens and if the lookup fails then no
exception message is delivered and the thread is suspended indefinitely.

The exception endpoint must have send and grant rights. Replying to the
exception message restarts the thread. For certain exception types, the contents of
the reply message may be used to set the values in the registers of the
thread being restarted.
See \autoref{sec:faults} for details.



\subsection{Message Layout of the Read-/Write-Registers Methods}
\label{sec:read_write_registers}

The registers of a thread can be read and written with the
\apifunc{seL4\_TCB\_ReadRegisters}{tcb_readregisters} and \apifunc{seL4\_TCB\_WriteRegisters}{tcb_writeregisters} methods. The register contents are transferred via the IPC buffer. The IPC buffer locations that registers are copied to/from are given below.

\ifxeightsix
\subsubsection{IA-32}

\begin{tabularx}{\textwidth}{p{0.5\textwidth}X}
\toprule
\textbf{Register} & \textbf{IPC Buffer location} \\
\midrule
\reg{EIP} & \ipcbloc{IPCBuffer[0]} \\
\reg{ESP} & \ipcbloc{IPCBuffer[1]} \\
\reg{EFLAGS} & \ipcbloc{IPCBuffer[2]} \\
\reg{EAX} & \ipcbloc{IPCBuffer[3]} \\
\reg{EBX} & \ipcbloc{IPCBuffer[4]} \\
\reg{ECX} & \ipcbloc{IPCBuffer[5]} \\
\reg{EDX} & \ipcbloc{IPCBuffer[6]} \\
\reg{ESI} & \ipcbloc{IPCBuffer[7]} \\
\reg{EDI} & \ipcbloc{IPCBuffer[8]} \\
\reg{EBP} & \ipcbloc{IPCBuffer[9]} \\
\reg{TLS\_BASE} & \ipcbloc{IPCBuffer[10]} \\
\reg{FS} & \ipcbloc{IPCBuffer[11]} \\
\reg{GS} & \ipcbloc{IPCBuffer[12]} \\
\bottomrule
\end{tabularx}
\fi

\subsubsection{ARM}

\begin{tabularx}{\textwidth}{p{0.5\textwidth}X}
\toprule
\textbf{Register} & \textbf{IPC Buffer location} \\
\midrule
\reg{PC} & \ipcbloc{IPCBuffer[0]} \\
\reg{SP} & \ipcbloc{IPCBuffer[1]} \\
\reg{CPSR} & \ipcbloc{IPCBuffer[2]} \\
\reg{R0-R1} & \ipcbloc{IPCBuffer[3-4]} \\
\reg{R8-R12} & \ipcbloc{IPCBuffer[5-9]} \\
\reg{R2-R7} & \ipcbloc{IPCBuffer[10-15]} \\
\reg{R14} & \ipcbloc{IPCBuffer[16]} \\
\bottomrule
\end{tabularx}


\section{Faults}
\label{sec:faults}

A thread's actions may result in a fault. Faults are delivered to the
thread's exception handler so that it can take the appropriate action.
The fault type is specified in the message label and is one of:
seL4\_CapFault, seL4\_VMFault, seL4\_UnknownSyscall, seL4\_UserException
or seL4\_Interrupt.

\subsection{Capability Faults}

Capability faults may occur in two places. Firstly, a capability fault
can occur when lookup of a capability referenced by a
\apifunc{seL4\_Call}{sel4_call} or \apifunc{seL4\_Send}{sel4_send} system call
failed (\apifunc{seL4\_NBSend}{sel4_nbsend} calls on
invalid capabilities silently fail). In this case, the capability
on which the fault occurred may be the capability being invoked or an
extra capability passed in the \texttt{caps} field in the IPC buffer.

Secondly, a capability fault can occur when \apifunc{seL4\_Recv}{sel4_recv} or \apifunc{seL4\_NBRecv}{sel4_nbrecv}
is called on a capability that does not exist, is not an endpoint or notification capability or does not have
receive permissions.

Replying to the fault IPC will restart the faulting thread. The contents of the
IPC message are given in \autoref{tbl:ipc_contents}.\\

\begin{table}[htb]
\noindent\begin{tabularx}{\textwidth}{XX}
\toprule
\textbf{Meaning} & \textbf{IPC buffer Location} \\
\midrule
Address at which to restart execution & \ipcbloc{IPCBuffer[0]} \\
Capability address & \ipcbloc{IPCBuffer[1]}\\
In receive phase (1 if the fault happened during a receive system call, 0
otherwise) & \ipcbloc{IPCBuffer[2]}\\
Lookup failure description. As described in \autoref{sec:lookup_fail_desc} &
\ipcbloc{IPCBuffer[3..]}\\
\bottomrule
\end{tabularx}
\caption{\label{tbl:ipc_contents}Contents of an IPC message.}
\end{table}

\subsection{Unknown Syscall}
\label{sec:unknown-syscall}

This fault occurs when a thread executes a system call with a syscall
number that is unknown to seL4.
The register set
of the faulting thread is passed to the thread's exception handler so that it
may, for example, emulate the system call if a thread is being
virtualised.

Replying to the fault IPC allows the thread to be restarted
and/or the thread's register set to be modified. If the reply has
a label of zero, the thread will be restarted. Additionally, if the
message length is non-zero, the faulting thread's register set will be
updated as shown in \autoref{tbl:unknown_syscall_result_arm} \ifxeightsix and
\autoref{tbl:unknown_syscall_result_ia32}\fi. In this case, the number of
registers updated is controlled with the length field of the message
tag.

\subsubsection{ARM}

\begin{table}[htb]
\begin{tabularx}{\textwidth}{XXX}
\toprule
\textbf{Value sent} & \textbf{Register set by reply} & \textbf{IPC buffer location} \\
\midrule
\reg{R0-R7} & (same) & \ipcbloc{IPCBuffer[0-7]} \\
\reg{FaultInstruction} & (same) & \ipcbloc{IPCBuffer[8]} \\
\reg{SP} & (same) & \ipcbloc{IPCBuffer[9]} \\
\reg{LR} & (same) & \ipcbloc{IPCBuffer[10]} \\
\reg{CPSR} & (same) & \ipcbloc{IPCBuffer[11]} \\
Syscall number & --- & \ipcbloc{IPCBuffer[12]} \\
\bottomrule
\end{tabularx}
\caption{\label{tbl:unknown_syscall_result_arm}Unknown system call outcome on
the ARM architecture.}
\end{table}

\ifxeightsix
\subsubsection{IA-32}
% FIXME: This table now reflows onto the following page with the paragraph after
% inserted here :(
\begin{table}[htb]
\begin{tabularx}{\textwidth}{XXX}
\toprule
\textbf{Value sent} & \textbf{Register set by reply} & \textbf{IPC buffer location} \\
\midrule
\reg{EAX} & (same) & \ipcbloc{IPCBuffer[0]} \\
\reg{EBX} & (same) & \ipcbloc{IPCBuffer[1]} \\
\reg{ECX} & (same) & \ipcbloc{IPCBuffer[2]} \\
\reg{EDX} & (same) & \ipcbloc{IPCBuffer[3]} \\
\reg{ESI} & (same) & \ipcbloc{IPCBuffer[4]} \\
\reg{EDI} & (same) & \ipcbloc{IPCBuffer[5]} \\
\reg{EBP} & (same) & \ipcbloc{IPCBuffer[6]} \\
\reg{EIP} & (same) & \ipcbloc{IPCBuffer[7]} \\
\reg{ESP} & (same) & \ipcbloc{IPCBuffer[8]} \\
\reg{EFLAGS} & (same) & \ipcbloc{IPCBuffer[9]} \\
Syscall number & --- & \ipcbloc{IPCBuffer[10]} \\
\bottomrule
\end{tabularx}
\caption{\label{tbl:unknown_syscall_result_ia32}Unknown system call outcome on
the IA-32 architecture.}
\end{table}
\fi


\subsection{User Exception}

User exceptions are used to deliver architecture-defined exceptions. For
example, such an exception could occur if a user thread attempted to
divide a number by zero.

Replying to the fault IPC allows the thread to be restarted
and/or the thread's register set to be modified. If the reply has
a label of zero, the thread will be restarted. Additionally, if the
message length is non-zero, the faulting thread's register set will be
updated as shown in \autoref{tbl:user_exception_result_arm} \ifxeightsix and
\autoref{tbl:user_exception_result_ia32}\fi. In this case, the number of
registers updated is controlled with the length field of the message
tag.

\subsubsection{ARM}

\begin{table}[htb]
\begin{tabularx}{\textwidth}{XXX}
\toprule
\textbf{Value sent} & \textbf{Register set by reply} & \textbf{IPC buffer location} \\
\midrule
\reg{FaultInstruction} & (same) & \ipcbloc{IPCBuffer[0]} \\
\reg{SP} & (same) & \ipcbloc{IPCBuffer[1]} \\
\reg{CPSR} & (same) & \ipcbloc{IPCBuffer[2]} \\
Exception number & --- & \ipcbloc{IPCBuffer[3]} \\
Exception code & --- & \ipcbloc{IPCBuffer[4]} \\
\bottomrule
\end{tabularx}
\caption{\label{tbl:user_exception_result_arm}User exception outcome on the ARM
architecture.}
\end{table}

\ifxeightsix
\subsubsection{IA-32}

\begin{table}[htb]
\begin{tabularx}{\textwidth}{XXX}
\toprule
\textbf{Value sent} & \textbf{Register set by reply} & \textbf{IPC buffer location} \\
\midrule
\reg{EIP} & (same) & \ipcbloc{IPCBuffer[0]} \\
\reg{ESP} & (same) & \ipcbloc{IPCBuffer[1]} \\
\reg{EFLAGS} & (same) & \ipcbloc{IPCBuffer[2]} \\
Exception number & --- & \ipcbloc{IPCBuffer[3]} \\
Exception code & --- & \ipcbloc{IPCBuffer[4]} \\
\bottomrule
\end{tabularx}
\caption{\label{tbl:user_exception_result_ia32}User exception outcome on the
IA-32 architecture.}
\end{table}
\fi

\subsection{VM Fault}
\label{sec:vm-fault}

The thread caused a page fault. Replying to the fault IPC will restart
the thread. The contents of the IPC message are given below.\\

% FIXME This table appears to be unified to make it architecture-independent,
% but all the other tables are broken down into ARM and IA-32, so this one
% should be as well for consistency.
\noindent\begin{tabularx}{\textwidth}{XX}
\toprule
\textbf{Meaning} & \textbf{IPC buffer location} \\
\midrule
Program counter to restart execution at. & \ipcbloc{IPCBuffer[0]} \\
Address that caused the fault. & \ipcbloc{IPCBuffer[1]} \\
Instruction fault (1 if the fault was caused by an instruction fetch). & \ipcbloc{IPCBuffer[2]}  \\
Fault status register (FSR). Contains information about the cause of the fault. Architecture dependent. & \ipcbloc{IPCBuffer[3]} \\
\bottomrule
\end{tabularx}\\ \\

\section{Domains}
\label{sec:domains}

Domains are used to isolate independent subsystems, so as to limit 
information flow between them.
The kernel switches between domains according to a fixed, time-triggered 
schedule.
The fixed schedule is compiled into the kernel via the constant 
\texttt{CONFIG\_NUM\_DOMAINS} and the global variable \texttt{ksDomSchedule}.

A thread belongs to exactly one domain, and will only run when that domain 
is active.
The \apifunc{seL4\_DomainSet\_Set}{domainset_set} method changes the domain 
of a thread.
The caller must possess a \obj{Domain} cap and the thread's \obj{TCB} cap.
The initial thread starts with a \obj{Domain} cap (see 
\autoref{sec:messageinfo}).


