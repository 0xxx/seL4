%
% Copyright 2014, General Dynamics C4 Systems
%
% This software may be distributed and modified according to the terms of
% the GNU General Public License version 2. Note that NO WARRANTY is provided.
% See "LICENSE_GPLv2.txt" for details.
%
% @TAG(GD_GPL)
%

\chapter{\label{ch:objects}Kernel Services and Objects}

A limited number of service primitives are provided by the microkernel;
more complex services may be implemented as applications on top of these
primitives. In this way, the functionality of the system can be extended
without increasing the code and complexity in privileged mode, while
still supporting a potentially wide number of services for varied
application domains.

The basic services seL4 provides are as follows:
\begin{description}
    \item[Threads] are an abstraction of CPU execution that supports
    running software;

    \item[Address spaces] are virtual memory spaces that each contain an
    application. Applications are limited to accessing memory in their
    address space;

    \item[Inter-process communication] (IPC) via \emph{endpoints} allows
    threads to communicate using message passing;

    \item[Device primitives] allow device drivers to be implemented as
    unprivileged applications.  The kernel exports hardware device
    interrupts via IPC messages; and

    \item[Capability spaces] store capabilities (i.e., access rights) to
    kernel services along with their book-keeping information.
\end{description}

This chapter gives an overview of these services, describes how kernel
objects are accessed by userspace applications, and describes how new
objects can be created.

\section{Capability-based Access Control}
\label{sec:cap-access-control}

The seL4 microkernel provides a capability-based access-control model.
Access control governs all kernel services; in order to perform an
operation, an application must \emph{invoke} a capability in its
possession that has sufficient access rights for the requested service.
With this, the system can be configured to isolate software components from
each other, and also to enable authorised, controlled communication
between components by selectively granting specific communication
capabilities.  This enables software-component isolation with a high
degree of assurance, as only those operations explicitly authorised by
capability possession are permitted.

A capability is an unforgeable token that references a specific kernel
object (such as a thread control block) and carries access rights that
control what methods may be invoked.
Conceptually, a capability resides in an application's \emph{capability
space}; an address in this space refers to a \emph{slot} which may or
may not contain a capability.  An application may refer to
a capability---to request a kernel service, for example---using the
address of the slot holding that capability.  This means, the seL4 
capability model is an instance of a \emph{segregated} (or \emph{partitioned})
capability system, where capabilities are managed by the kernel.

Capability spaces are implemented as a directed graph of kernel-managed
\emph{capability nodes} (\obj{CNode}s).  A \obj{CNode} is a table of
slots, where each slot may contain further \obj{CNode} capabilities. An
address in a capability space is then the concatenation of the indices
of the \obj{CNode} capabilities forming the path to the destination
slot; we discuss \obj{CNode} objects in detail in \autoref{ch:cspace}.

Capabilities can be copied and moved within capability spaces, and
also sent via IPC. This allows creation of applications with specific
access rights, the delegation of authority to another application, and
passing to an application authority to a newly created (or selected)
kernel service. Furthermore, capabilities can be \emph{minted} to
create a derived capability with a subset of the rights of the
original capability (never with more rights). A newly minted
capability can be used for partial delegation of authority.

Capabilities can also be revoked to withdraw
authority. Revocation recursively removes any capabilities that have
been derived from the original capability being revoked. The propagation of
capabilities through the system is controlled by a
\emph{take-grant}-based model~\cite{Elkaduwe_GE_08,Boyton_09}.

\section{System Calls}
\label{sec:syscalls}
\label{sec:sys_send}
\label{sec:sys_wait}
\label{sec:sys_call}
\label{sec:sys_reply}
\label{sec:sys_nbsend}
\label{sec:sys_replywait}
\label{sec:sys_yield}

The seL4 kernel provides a message-passing service for communication between
threads. This mechanism is also used for communication with kernel-provided
services. There is a standard message format, each message containing a
number of data words and possibly some capabilities. The structure and encoding
of these messages are described in detail in \autoref{ch:ipc}. 

Threads send messages by invoking capabilities within their capability space.
When an endpoint capability is invoked in this way the message will be
transferred through the kernel to another thread. When capabilities to kernel
objects are invoked, the message will be interpreted as a method invocation in a
manner specific to the type of kernel object. For example, invoking a thread
control block (TCB) capability with a correctly formatted message will suspend
the target thread.

The kernel provides the following system calls:
\begin{description}
    \item[\apifunc{seL4\_Send}{sel4_send}] delivers a message through the named
    capability and then allows the application to continue. If the invoked
    capability is an endpoint and no receiver is ready to receive the message
    immediately, the sending thread will be blocked until the message can be
    delivered. No error code or response will be returned by the receiving
    thread or kernel object.

    \item[\apifunc{seL4\_NBSend}{sel4_nbsend}] performs a non-blocking send. It
    is similar to \apifunc{seL4\_Send}{sel4_send} except that if the message
    cannot be received immediately, the message is silently dropped. Like
    \apifunc{seL4\_Send}{sel4_send}, no error code or response will be returned
    by the object.

    \item[\apifunc{seL4\_Call}{sel4_call}] is a \apifunc{seL4\_Send}{sel4_send}
    that blocks the sending thread until a reply message is received. When the
    sent message is delivered to another thread (via an \obj{Endpoint}), an
    additional `\emph{reply}' capability is added to the message and delivered
    to the receiver to give it the right to reply to the sender. The reply
    capability is stored in a special purpose slot in the receiver's \obj{TCB}.
    When invoking capabilities to kernel services, using
    \apifunc{seL4\_Call}{sel4_call} allows the kernel to return an error code
    or other response within a reply message.

    \item[\apifunc{seL4\_Wait}{sel4_wait}] is used by a thread to receive
    messages through endpoints. If no message is ready to be sent, the thread
    will block until a message is sent. This system call works only on
    endpoint capabilities, raising a fault (see section \ref{sec:faults}) when
    attempted with other capability types.

    \item[\apifunc{seL4\_Reply}{sel4_reply}] is used to respond to a
    \apifunc{seL4\_Call}{sel4_call}, using the capability generated by the
    \apifunc{seL4\_Call}{sel4_call} system call and stored in the replying
    thread's TCB. It delivers the message to the calling thread, waking it in
    the process.

    There is space for only one reply capability in each thread's TCB, so the
    \apifunc{seL4\_Reply}{sel4_reply} syscall can be used to reply to the most
    recent caller only. The \apifunc{seL4\_CNode\_SaveCaller}{cnode_savecaller}
    method that will be described later can be used to save the reply
    capability into regular capability space, where it can be used with
    \apifunc{seL4\_Send}{sel4_send}.

    \item[\apifunc{seL4\_ReplyWait}{sel4_replywait}] is a combination of \apifunc{seL4\_Reply}{sel4_reply} and
    \apifunc{seL4\_Wait}{sel4_wait}. It exists for efficiency reasons: the common case of
    replying to a request and waiting for the next can be performed in
    a single kernel system call instead of two.

    \item[\apifunc{seL4\_Yield}{sel4_yield}] is the only system call that does not require
    a capability to be used. It causes the calling thread to give up the
    remainder of its timeslice to another runnable thread of the
    same priority level. If there are no runnable threads with the same
    priority, the calling thread is resumed and the system call has no
    effect.
\end{description}

\section{Kernel Objects}
\label{s:sel4_internals}

In this section we give a brief overview of the kernel-implemented
object types whose instances (also simply called \emph{objects}) can be invoked by applications. The interface to these
objects forms the interface to the kernel itself. The creation and use
of the high-level kernel services is achieved by the creation,
manipulation, and combination of these kernel objects:

\begin{description}

    \item[\obj{CNodes}] (see \autoref{ch:cspace}) store capabilities, giving threads permission to
    invoke methods on particular objects.
    Each \obj{CNode} has a fixed number of slots,
    always a power of two, determined when the \obj{CNode} is created. Slots
    can be empty or contain a capability.

    \item[Thread Control Blocks] (\obj{TCB}s; see \autoref{ch:threads}) represent a thread of
    execution in seL4. Threads are the unit of execution that is
    scheduled, blocked, unblocked, etc., depending on the application's
    interaction with other threads.

    \item[IPC Endpoints] (see \autoref{ch:ipc}) may be used to facilitate interprocess
    communication between threads. The seL4 microkernel supports two
    types of endpoints:
    \begin{itemize}
        \item \emph{Synchronous} endpoints (\obj{Endpoint}),
        which cause the sending thread to block until its message
        is received; and

        \item \emph{Asynchronous} endpoints (\obj{AsyncEP}),
        which only allow short messages to be sent, but do not
        cause the sender to block.
    \end{itemize}

    A capability to an endpoint can be restricted to be
    send-only or receive-only. Additionally, capabilities to \obj{Endpoint}
    objects may be configured to allow capabilities to be transferred across to
    other threads.

    \item[Virtual Address Space Objects] (see \autoref{ch:vspace}) 
    are used to construct a virtual
    address space (or VSpace) for one or more threads. These
    objects largely directly correspond to those of the hardware, and
    as such are architecture-dependent. The kernel also includes \obj{ASID
    Pool} and \obj{ASID Control} objects for tracking the status of
    address spaces.

    \item[Interrupt Objects] (see \autoref{ch:io}) give applications the ability to receive
    and acknowledge interrupts from hardware devices.
    Initially, there is a capability to \obj{IRQControl},
    which allows for the creation of \obj{IRQHandler} capabilities.
    An \obj{IRQHandler} capability permits the management of a specific 
    interrupt source associated with a specific device.
    It is delegated to
    a device driver to access an interrupt source. The \obj{IRQHandler}
    object allows threads to wait for and acknowledge individual
    interrupts.

    \item[Untyped Memory] (see \autoref{sec:kernmemalloc}) is the foundation of memory allocation
    in the seL4 kernel.  Untyped memory capabilities have a single method
    which allows the creation of new kernel objects. If the method
    succeeds, the calling thread gains access to capabilities to the
    newly-created objects. Additionally, untyped memory objects can be
    divided into a group of smaller untyped memory objects allowing
    delegation of part (or all) of the system's memory.  We discuss
    memory management in general in the following sections.

\end{description}

\section{Kernel Memory Allocation}
\label{sec:kernmemalloc}

The seL4 microkernel does not dynamically allocate memory for kernel objects.
Instead, objects must be explicitly created from application-controlled memory
regions via \obj{Untyped Memory} capabilities.  Applications must have
explicit authority to memory (through these \obj{Untyped Memory} capabilities) in
order to create new objects, and all objects consume a fixed amount of memory once
created. These mechanisms can be used to precisely control
the specific amount of physical memory available to applications,
including being able to enforce isolation of physical memory access
between applications or a device.  There are no arbitrary resource
limits in the kernel apart from those dictated by the
hardware\footnote{The treatment of virtual ASIDs imposes a fixed number
of address spaces. This limitation is to be removed in future
versions of seL4.}, and so many denial-of-service attacks via resource
exhaustion are avoided.

At boot time, seL4 pre-allocates the memory required for the kernel
itself, including the code, data, and stack sections (seL4 is a single
kernel-stack operating system). It then creates an initial user
thread (with an appropriate address and capability space).
The kernel than hands all remaining memory to
the initial thread in the form of capabilities to \obj{Untyped Memory}, and
some additional capabilities to kernel objects that were required to
bootstrap the initial thread.  These \obj{Untyped Memory} regions can then be split into
smaller regions or other kernel objects using the
\apifunc{seL4\_Untyped\_Retype}{untyped_retype} method; the created objects are termed \emph{children} of
the original untyped memory object.

The user-level application that creates an object using \apifunc{seL4\_Untyped\_Retype}{untyped_retype}
receives full authority over the resulting object. It can then delegate
all or part of the authority it possesses over this object to one or
more of its clients.

\subsection{Reusing Memory}
\label{s:memRevoke}

The model described thus far is sufficient for applications to
allocate kernel objects, distribute authority among client
applications, and obtain various kernel services provided by these
objects.  This alone is sufficient for a simple static system
configuration.

The seL4 kernel also allows \obj{Untyped Memory} regions to be reused.
Reusing a region of memory is allowed only
when there are no dangling references (i.e., capabilities) left to the
objects inside that memory.  The kernel tracks
\emph{capability derivations}, i.e., the children generated by the
methods \apifunc{seL4\_Untyped\_Retype}{untyped_retype}, \apifunc{seL4\_CNode\_Mint}{cnode_mint}, \apifunc{seL4\_CNode\_Copy}{cnode_copy}, and
\apifunc{seL4\_CNode\_Mutate}{cnode_mutate}.

The tree structure so generated is termed the \emph{capability
derivation tree} (CDT).\footnote{Although the CDT conceptually is a separate
data structure, it is implemented as part of the CNode object and so
requires no additional kernel meta-data.}  For example, when a user
creates new kernel objects by retyping untyped memory, the newly created
capabilities would be inserted into the CDT as children of the untyped
memory capability.

For each \obj{Untyped Memory} region, the kernel keeps
a \emph{watermark} recording how much of the region has previously been
allocated. Whenever a user requests the kernel to create new objects in
an untyped memory region, the kernel will carry out one of two actions:
if there are already existing objects allocated in the region, the
kernel will allocate the new objects at the current watermark level, and
increase the watermark. If all objects previously allocated in the
region have been deleted, the kernel will reset the watermark and start
allocating new objects from the beginning of the region again.

Finally, the \apifunc{seL4\_CNode\_Revoke}{cnode_revoke} method provided by \obj{CNode} objects
destroys all capabilities derived from the argument capability. Revoking
the last capability to a kernel object triggers the \emph{destroy}
operation on the now unreferenced object. This simply cleans up any in-kernel dependencies between
it, other objects and the kernel.

By calling \apifunc{seL4\_CNode\_Revoke}{cnode_revoke} on the original capability to an untyped memory
object, the user removes all of the untyped memory object's
children---that is, all capabilities pointing to objects in the untyped
memory region.  Thus, after this invocation there are no valid references
to any object within the untyped region, and the region may be safely
retyped and reused.

